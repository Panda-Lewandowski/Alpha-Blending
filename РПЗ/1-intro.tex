\Introduction

 Оптимизация алгоритма - один из важных этапов разработки программного обеспечения. Модификации ПО чаще всего направлены на улучшение выходных характеристик алгоритмов при тех же технических требованиях. Напротив, изменения  продукта в визуальном плане составляют, пожалуй, меньшую долю всех модификаций. Острую необходимость в оптимизации  требуют графические редакторы и компьютерные игры. В них основные вычислительные затраты берут на себя сложные алгоритмы компьютерной графики. К примеру, серьезным недостатком метода трассировки лучей является производительность, так как для каждого пикселя необходимо заново производить процесс определения цвета, рассматривая каждый луч наблюдения в отдельности.
 
Можно выделить четыре вектора оптимизации алгоритма: 
\begin{enumerate}
\item улучшение временных характеристик (уменьшение тактов процессора, требующихся для выполнения задачи). К примеру, минимизация операций деления или вычисления корня. 
\item качественных характеристик. Например, увеличение точности вычислений при дифференцировании с помощью формул Рунге, имеющих высокий порядок точности или достижение более качественного решения в задачах классификации  при использовании машинного обучения.
\item требуемых ресурсов (уменьшение пространственной сложности алгоритма). Однако, известно несколько примеров, когда эффективные алгоритмы требуют таких больших объемов машинной памяти (без возможности использования более медленных внешних средств хранения), что этот фактор сводит на нет преимущество «эффективности» алгоритма. Примером такого алгоритма является алгоритм Евклида для нахождения НОД двух целых чисел.
\item устойчивости алгоритма (уменьшение чувствительности алгоритма к изменениям входных данных).
\end{enumerate}

В компьютерной графике немаловажное значение имеют временные характеристики и затраты по памяти. Начинать модификации ПО стоит с оптимизации более простых и фундаментальных алгоритмов. К таким алгоритмам относят операцию альфа-смешения двух пикселей. Она является одной из основных операций в графических редакторах. Альфа-смешение применяется при слиянии двух слоев рисования, наложении друг на друга нескольких  примитивов и использовании масок. Уменьшение памяти, используемое для хранения двух пикселей, как правило, произвести невозможно, т.к. в большинстве цифровых пространств оно является минимальным и конечным. Оптимизация данной операции по времени даст уменьшение времени работы сложных алгоритмов.  Цель данного курсового проекта -- добиться максимально быстрой работы альфа-смешения в наиболее используемом
цветовом пространстве с использованием технологий оптимизации вычислений.
 
  
 
 
 
