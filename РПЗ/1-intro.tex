\Introduction

 Оптимизация алгоритма - один из самых важных этапов разработки программного обеспечения. Он необходим практически при создании любого программного продукта. Модификации ПО, как правило, направлены на улучшение выходных характеристик алгоритмов при тех же технических требованиях. Напротив, изменения  продукта в визуальном плане составляют, пожалуй, наименьшую долю всех модификаций. Острую необходимость в оптимизации  требуют графические редакторы и игры. В них основные вычислительные затраты берут на себя сложные алгоритмы компьютерной графики. К примеру, серьезным недостатком метода трессировки лучей является производительность, так как для каждого пикселя необходимо заново производить процесс определения цвета, рассматривая каждый луч наблюдения в отдельности.
 
Можно выделить три вектора оптимизации алгоритма: 
\begin{enumerate}
\item улучшение временных характеристик (уменьшение тактов процессора, требующихся для выполнения данной задачи). К примеру, минимизация операций деления и вычисления корня. 
\item качественных характеристик. Например, увеличение точности вычислений
\item требуемых ресурсов. (пример про оперативную память). 
\end{enumerate}
 %(немножко про аппаратную часть тип сервера и тд )
 
 Улучшение временных характеристик с прежним аппаратным обеспечением, и блаблабла
 
 
 
 
