\chapter{ Аналитический раздел}
\label{cha:analysis}
\section{Постановка задачи}
В соответствии с техническим заданием на курсовой проект необходимо реализовать программный модуль для сравнения различных алгоритмов смешения цветов в пространстве RGBA. Внедрение оптимизированного алгоритма будет производиться в открытый графический редактор KDE Krita, имеющий несколько реализаций для работы с цветом. Понадобится разобрать несколько ключевых моментов в теории цвета: концептуальные представления цвета и  математику работы с цветом относительно его представления, а также технологии оптимизации вычислений.

\section {Цветовые пространства и модели}
Цветовая модель -- это способ представления цвета в виде кортежа некоторых его независимых между собой характеристик. Как правило, это три составляющие его компоненты, например красный, зелёный и синий или тон, насыщенность и яркость. Цветовое пространство же -- представление цветового множества с помощью такого координатного пространства, что каждая ось преставляет возможные значения одной из компонент кортежа из цветовой модели. Необходимо четко различать цветовые модели и цветовые координатные системы: в первом случае речь идет о способе воспроизведения цветовых ощущений, а во втором — об измерении этих ощущений.

\subsection{CIE RGB}
$RGB$ -- это аддитивная цветовая модель, в которой красный, зеленый и синий свет суммируются различными способами для воспроизведения широкого спектра цветов. Название модели происходит от инициалов трех аддитивных первичных цветов, красного $R$ , зеленого $G$ и синего $B$.
С помощью этой модели цвет можно представить в виде триплета чисел от 0 до определенного максимального значения, соответственно представляющих долю основных красного, зеленого и синего цветов. Если все компоненты равны нулю, результатом будет черный цвет; если все находятся на максимуме, результат - самый яркий представляемый белый.
Эти диапазоны можно количественно определить несколькими способами:
\begin{enumerate}
	\item От 0 до 1, с любым дробным значением между ними. Это представление используется в теоретических анализах и в системах, которые используют представления с плавающей точкой.
	\item Каждое значение цветового компонента также может быть записано в процентах от 0\% до 100\%.
	\item В компьютерах значения компонентов часто хранятся как целые числа в диапазоне от 0 до 255, диапазон, который может предложить один 8-разрядный байт. Они часто представлены как десятичные или шестнадцатеричные числа.
	\item Высококачественное цифровое графическое оборудование часто может иметь дело с большими целыми диапазонами для каждого основного цвета, например 0..1023 (10 бит), 0..65535 (16 бит) или даже больше, путем расширения 24-бит ( три 8-битных значения) до 32-разрядных , 48-битных или 64-битных единиц (более или менее независимых от размера слова конкретного компьютера )
\end{enumerate}	

\begin{figure}[ht!]
	\centering{ 
		\includegraphics[width=0.4\textwidth]{img/2_bit.png}
		\includegraphics[width=0.4\textwidth]{img/32_bit.png}
		\caption{Cравнение изображений с глубиной цвета 2 бита (слева) и 32 бита (справа)}}
\end{figure}

Количество бит (объём памяти), используемое для хранения и представления цвета при кодировании одного пикселя называют глубиной цвета. 

В начале прошлого века Международная Комиссия по освещению (CIE —
Communication Internationale de l`Eclairage) предприняла попытку  измерить и систематизировать цветовые ощущения человека, вызываемые спектрально-чистыми
цветами, расположенными на всем протяжении видимого спектра: от фиолетового до
красного. Результатом этого эксперимента и стала модель RGB. 

\begin{figure}[ht!]
	\centering{ 
		\includegraphics[width=0.4\textwidth]{img/img1.png}
		\includegraphics[width=0.4\textwidth]{img/img2.png}
		\caption{Представление цветового куба в пространстве  $RGB$ }}
\end{figure}

\subsection{HSL и HSV}
HSL и HSV являются двумя наиболее распространенными представлениями о цилиндрических или конусных координатах точек в цветовой модели RGB. 

HSL, HLS или HSI  — цветовая модель, в которой цветовыми координатами являются тон (Hue), насыщенность (Saturation) и светлота(Lightness/Intensity). В HSV или HSB светлота заменяется яркостью (Brightness) или значением цвета(Value).

\begin{figure}[ht!]
	\centering{ 
		\includegraphics[width=0.4\textwidth]{img/HSV_cone.png}
		\includegraphics[width=0.4\textwidth]{img/HSV_cylinder.png}
		\caption{Коническое и цилиндрическое представление модели HSV.}}
\end{figure}

\begin{figure}[ht!]
	\centering{ 
		\includegraphics[width=0.4\textwidth]{img/HSL_cone.png}
		\includegraphics[width=0.4\textwidth]{img/HSL_cylinder.png}
		\caption{Коническое и цилиндрическое представление модели HSL.}}
\end{figure}

HSL и HSV представляют собой цилиндр, где оттенок или тон есть угол, начиная с красного первичного при 0°, проходя через зеленый первичный при 120° и синий первичный при 240°, а затем возвращаясь назад на красный при 360°. Центральная вертикальная ось содержит нейтральные , ахроматические или серые цвета, начиная с черного цвета при освещенности 0 или значения (Value) 0 внизу до белого при яркости 1 или значении (Value) 1, вверху.

В обоих циллиндрах, первичные и вторичные цвета -- красный, желтый , зеленый, голубой , синий и пурпурный  -- и линейные смеси между соседними парами таких цветов, иногда называемые чистыми цветами , расположены вокруг внешнего края цилиндра с насыщенностью 1. Эти насыщенные цвета имеют яркость ½ в HSL, тогда как в HSV они имеют значение 1. Смешивание этих чистых цветов с черными  производными оттенками - не изменяют насыщенности. В HSL насыщенность также не изменяется при тонировании белым, и только смеси и с черным и с белым оттенками имеют насыщенность менее 1. В HSV только тонирование снижает насыщенность.

\begin{figure}[ht!]
	\centering{ 
		\includegraphics[width=0.4\textwidth]{img/HSV-Slider.png}
		\includegraphics[width=0.4\textwidth]{img/Triangulo_HSV.png}
		\caption{Представление HSV в прикладном ПО.}}
\end{figure}


Оба эти представления широко используются в компьютерной графике, они часто более удобны, чем RGB, но оба они также подвергаются критике за неадекватное разделение атрибутов цветопередачи или отсутствие единообразия восприятия. Перевод из HSV/HSL в RGB довольно сложен. Если  необходимо получить результат смешения двух цветов в одном из этих двух форматов, то лучше произвести все вычисления в RGB, а затем перевести в HSL/HSV по данным формулам: 

RGB → HSV: 

\begin{equation}
 H \in [0°, 360°] 
 \end{equation}
 \begin{equation}
 S,V,R,G,B \in [0,1]
\end{equation}
Пусть $MAX$ — максимальное значение из $R$, $G$ и $B$, а $MIN$ — минимальное из них. Тогда
\begin{equation}
H={\begin{cases} 0°,        if ~ MAX=MIN \\
	60°\times {\frac  {G-B}{MAX-MIN}}+0°, if~MAX=R, G\geq B \\
	60°\times {\frac  {G-B}{MAX-MIN}}+360°, if~MAX=R, G < B \\
	60°\times {\frac  {B-R}{MAX-MIN}}+120°, if~MAX=G \\
	60°\times {\frac  {R-G}{MAX-MIN}}+240°, if~ MAX= B \\
	\end{cases}}
\end{equation}

 \begin{equation}
S={\begin{cases} 0, if~{\displaystyle MAX=0} \\
else~{\displaystyle 1-{\dfrac {MIN}{MAX}}}  
	\end{cases}}	
\end{equation}
 \begin{equation}
V=MAX
\end{equation}

RGB → HSL:\\
Пусть $MAX$ — максимальное значение из $R$, $G$ и $B$, а $MIN$ — минимальное из них. Тогда $H$ определяется аналогично модели HSV.
 \begin{equation}
L=\frac{1}{2}(MAX+MIN)
\end{equation}
 \begin{equation}
S={\begin{cases} 0, if  L = 0 \vee MAX = MIN \\
	 \frac{MAX-MIN}{1-|1-(MAX+MIN)|}
	\end{cases}}
\end{equation}

\section{Методы смешивания цветов}
 Смешение цветов - это синтез нового цвета на основе двух других.  Видимые в естественных условиях цвета, как правило, являются результатом
 смешения спектральных цветов.
 Существует два различных типа смешения цветов. Это аддитивное (слагательное) смешение и субтрактивное (вычитательное) смешение.
 
\subsection{Аддитивный синтез}
Аддитивный цвет - это цвет, созданный путем смешивания нескольких различных цветов света, причем оттенки красного , зеленого и синего являются наиболее распространенными основными цветами, используемыми в аддитивной цветовой системе.Автором теории аддитивного синтеза считают Джеймса Клерка Максвелла, вдохновленного теории Юнга-Гельмгольца о трехцветном цветовом зрении.

\begin{figure}[ht!]
	\centering{ 
		\includegraphics[width=0.45\textwidth]{img/AdditiveColor.png}
		\caption{Аддитивный синтез: первичные красный, зеленый и синий в смешении попарно дают вторичные цвета, а все три дают белый цвет}}
\end{figure}

Существует большая разница между чистым спектральным желтым светом с длиной волны около 580 нм и смесью красного и зеленого света. Тем не менее, оба стимулируют наши глаза подобным образом, поэтому мы не обнаруживаем эту разницу, и оба являются желтым светом для человеческого глаза. 


Компьютерные мониторы и телевизоры являются наиболее распространенными примерами аддитивного синтеза. Каждый пиксель на большинстве типов цветных видеодисплеев состоит из красных, зеленых и синих субпикселей, свет из которых сочетается в разных пропорциях, чтобы производить все остальные цвета, а также белые и оттенки серого. Цветные субпикселы не перекрываются на экране, но при просмотре даже с небольшого  расстояния они перекрываются и смешиваются на сетчаткой глаза, производя тот же результат, что и внешнее наложение.


\subsection{Субтрактивный синтез}
Субтрактивный синтез объясняет смешение ограниченного набора красителей, красок, пигментов или натуральных красителей для создания более широкого диапазона цветов, каждый из которых является результатом частичного или полного вычитания (то есть, поглощения) некоторых длин волн света. Цвет, который отображается на поверхности, зависит от того, какие части видимого спектра не поглощаются и ,следовательно , остаются видимыми.

\begin{figure}[ht!]
	\centering{ 
		\includegraphics[width=0.45\textwidth]{img/SubtractiveColor.png}
		\caption{Субтрактивный синтез, где роль первичных цветов играют   голубой(cyan), пурпурный(magenta) и жёлтый.}}
\end{figure}

\section{Каналы}
Каналом в контексте цифровой модели можно назвать изображение в оттенках серого того же размера, что и цветное, выполненное только из одного компонента основных цветов. Соответственно RGB имеет три канала: красный, зеленый и голубой. Если изображение RGB 24-разрядное (промышленный стандарт по состоянию на 2005 год), каждый канал имеет 8 бит, для красного, зеленого и синего - другими словами, изображение состоит из трех (по одному для каждого канала), где каждое может хранить дискретные пиксели с обычными интенсивностями яркости между 0 и 255. 

\subsection{Альфа-канал}
Термин «альфа-канал» впервые введён в оборот Алви Смитом в конце 1970-х гг. и детально проработан в статье Томаса Портера и Тома Даффа 1984 года.

Альфа-каналом назовём компонету цветовой модели, представляющую коэффициент смешивания для управления линейной интерполяцией цветов переднего плана и фона. Такую компоненту будем обозначать $\alpha$, $\alpha = 0$ будем сопоставлять с полной прозрачностью пикселя, а $\alpha=1 (255)$ -- с полной непрозрачностью. \cite{bib1} 

Если в изображении используется альфа-канал, доступны два общих представления: прямая (непривязанная) $\alpha$ и премультиплексированная (ассоциированная) $\alpha$. С прямой $\alpha$ компоненты RGB представляют цвет объекта или пикселя, не обращая внимания на его непрозрачность. С премультиплексированной $\alpha$ компоненты RGB представляют цвет объекта или пикселя, скорректированный на его непрозрачность путем умножения. Такое представление обладает рядом преимуществ: 
\begin{enumerate}
	\item премультиплексированное $\alpha$-смешивание является ассоциативным.
	\item интерполяция и фильтрация дают правильные результаты. При интерполяции или фильтрации изображений без предварительно умноженной $\alpha$ с резкими границами между прозрачными и непрозрачными областями  может привести к границам цветов, которые не были видны в исходном изображении. Ошибки также возникают в областях полупрозрачности, потому что компоненты RGB неправильно взвешены, что приводит к некорректному взвешиванию цвета более прозрачных пикселей.
	\item уникальное представление для прозрачных пикселей. Представления цветов в виде (1, 0.5, 1, 0) невозможны.
\end{enumerate}


\subsection{RGBA}
RGBA -- цветовое пространство RGB c альфа-каналом. Четверка $(R, G, B, \alpha)$ говорит о том, что пиксель покрыт цветом $(\alpha R, \alpha G, \alpha B)$. 

\begin{figure}[ht!]
	\centering{ 
		\includegraphics[width=0.5\textwidth]{img/img6.png}
		\includegraphics[width=0.45\textwidth]{img/img7.png}
		\caption{Альфа-канал первого изображения меняется от 1 до нуля сверху вниз. На втором изображении совмещены изображения маяка и предыдущее изображение.}}
\end{figure}

Важно различать два ключевых пиксельных представления:
непрозрачный черный  (0,0,0,1) и полностью прозрачный пиксел  (0,0,0,0).


\section{Альфа-смешение}
Альфа-смешивание -- это смешивание двух и более цветов с учетом их альфа-каналов. Альфа-смешивание представляет собой выпуклую комбинацию из двух цветов, обеспечивающую эффект прозрачности в компьютерной графике. Стоит отметить несколько граничных случаев. Если цвет переднего плана полностью прозрачный, смешанный цвет будет цветом фона. И наоборот, если он полностью непрозрачен, смешанный цвет будет цветом переднего плана.
Прозрачность может варьироваться, и в этом случае смешанный цвет вычисляется как средневзвешенное значение цветов переднего плана и фона.

\subsection{Расчёт результирующего цвета}
Пусть пиксель А имеет $\alpha = \alpha_{A}$ и "чистый" цвет $A$, а пиксель B  -- $\alpha = \alpha_{B}$ и "чистый" цвет $B$. Таким образом,  результирующий цвет каждого отдельного взятого пикселя будет равен

\begin{equation}
C_{A} = \alpha_{A}A 
\end{equation}
\begin{equation}
C_{B} = \alpha_{B}B
\end{equation}

Если асссоциировать $\alpha$ с процентным покрытием пикселя равномерным пикселем \cite{bib1}, то пиксель А может покрыть своей непрозрачной составляющей не более, чем $1- \alpha_{B}$ процентов пикселя.  Отсюда следует, что результирующий цвет таков

\begin{equation}
C_{O} = \alpha_{B}B + (1- \alpha_{B})\alpha_{A}A = C_{B} + (1- \alpha_{B})C_{A},
\end{equation}
что является выпуклой комбинацией A и B.
Стоит заметить, что данный результат может оказаться неправильным в случае, если характеристики  А и В совпадают.  

Можно вывести формулу (1.10) более строго. 

\subsection{Аналитический вывод результирующего цвета}
Обозначим смешение цветов $A$ и $B$ как $A \oplus B$. 
Первое предположение состоит в том, что в случае, когда фон непрозрачен (т.е. $\alpha _ {B} = 1$), Оператор $\oplus$ представляет собой выпуклую комбинацию из A и B:
\begin{equation}
O= \alpha_{A}A + (1- \alpha_{A})B
\end{equation}
Второе предположение состоит в том, что оператор должен удовлетворять ассоциативному правилу:
\begin{equation}
(A \oplus B) \oplus C = A \oplus (B \oplus C)
\end{equation}
Пусть A и B имееют переменную прозрачность, а C непрозрачен. Тогда нужно найти
\begin{equation}
O = A \oplus B
\end{equation}
Из (1.12) и (1.13):
\begin{equation}
O \oplus C = A \oplus (B \oplus C)
\end{equation}
Т.к. С непрозрачен, то и В $\oplus$ C непрозрачен. , поэтому в приведенном выше уравнении каждый  $\oplus$ оператор можно записать в виде выпуклой комбинации:
\begin{equation}
\alpha_{O}O + (1 - \alpha_{O}) C = \alpha_{A}A + (1 - \alpha_{A})(\alpha_{B} + (1 - \alpha_{B})C) \\
= \alpha_{A}A + (1 - \alpha_{A})\alpha_{B}B + (1-\alpha_{A})(1-\alpha_{B})C
\end{equation}
Отсюда видно, что это представляет собой уравнение вида $ X_{0} + Y_{0} C = X_{1} + Y_{1} C$, установив  $X_{0} = X_{1}$, а также $Y_{0} = Y_ {1}$ мы получаем

\begin{equation}
\alpha_{O} = 1 - (1 - \alpha_{A})(1 - \alpha_{B}) =  \alpha_{A} + \alpha_{B}(1-\alpha_{A})
\end{equation}

\begin{equation}
O = \frac{\alpha_{A}A + (1-\alpha_{A})\alpha_{B}B}{\alpha_{O}}
\end{equation}

Интересно также отметить, что оператор  $\oplus$ удовлетворяет всем требованиям некоммутативного моноида, где единичный элемент $е$ выбирается таким образом, что  $e\oplus A = A \oplus e = A$. Т.е. единичный элемент может быть любым кортежем $\langle C, \alpha \rangle$ c $\alpha = 0$.

Если же используется премультиплексированная $\alpha$, то уравнение (1.18) примет вид, аналогичный (1.10): 

\begin{equation}
C_{O} = C_{A} + (1-\alpha_{A})C_{B}
\end{equation}

\subsection{Сравнение премультиплексированной  $\alpha$ и прямой $\alpha$}
Премультиплексированное представление обладает рядом преимуществ: 
\begin{enumerate}
	\item премультиплексированное $\alpha$-смешивание является ассоциативным.
	\item интерполяция и фильтрация дают правильные результаты. При интерполяции или фильтрации изображений без предварительно умноженной $\alpha$ с резкими границами между прозрачными и непрозрачными областями  может привести к границам цветов, которые не были видны в исходном изображении. Ошибки также возникают в областях полупрозрачности, потому что компоненты RGB неправильно взвешены, что приводит к некорректному взвешиванию цвета более прозрачных пикселей.
	\item уникальное представление для прозрачных пикселей. Представления цветов в виде (1, 0.5, 1, 0) невозможны.
\end{enumerate}
Использование же прямой $\alpha$ создает ряд проблем. Рассмотрим их подробнее.
 
\subsection{Проблема повторной композиции и \\ накопление погрешности}
Данная проблема проявляется при композиции/смешивании трех и более цветов, используя результат предыдущих двух цветов.

Пусть пиксель K является результатом для  J $\oplus$ I. Предположим, что мы хотим выполнить L $\oplus$ K, используя прямую $\alpha$.

\begin{equation}
K = \frac{\alpha_{J}J + (1-\alpha_{J})\alpha_{I}I}{\alpha_{K}}
\end{equation}

\begin{equation}
L \oplus K = \frac{\alpha_{L}L + (1-\alpha_{L})\alpha_{K}K}{\alpha_{L \oplus K}}
\end{equation}

Как видно из вышепредставленных формул, мы все время делим и умножаем на $\alpha_{K}$, что сильно снижает производительность и эффективность. При использовании целочисленного представления цветов это приводит к большим неточностям при даже небольшом наслаивании пикселей. Также при $\alpha = 0$ это приводит к ошибке деления на ноль. Сама $\alpha = 0$ приводит к неоднозначности: пиксель (1, 1, 1, 0) может существовать так же, как и (1, 1, 1, 0). Во-первых, это лишено физического смысла. Во-вторых, при смешивании это может приводить к лишним вычислениям, когда пиксель не будет вносить вклад в результирующий цвет, но будет просчитан, т.к. будет обладать цветом. 

\subsection{Конкретизация формул для альфа-смешения}
Пусть $src$ смешивается с $dst$. Результирующий цвет обозначим как $out$.
Тогда, используя прямую  $\alpha$:
\begin{equation}
\begin{cases} \alpha_{out}= \alpha_{src} + \alpha_{dst}(1- \alpha_{src}) \\
out_{RGB} = \frac{(src_{RGB}\alpha_{src} + dst_{RGB}\alpha_{dst}(1-\alpha_{src}))} {\alpha_{out}}
\end{cases}
\end{equation}

И премультиплексированную  $\alpha$:
\begin{equation}
\begin{cases} 
\alpha_{out}= \alpha_{src} + \alpha_{dst}(1- \alpha_{src}) \\
out_{RGB} = src_{RGB}\alpha_{src} + dst_{RGB}(1 - \alpha_{src})
\end{cases}
\end{equation}

Также формулу (1.22) после упрощения через деление на 255 (при использовании RGBA8)  можно представить в виде:
\begin{equation}
\begin{cases} 
\alpha_{out}= \alpha_{src} + \alpha_{dst}(1- \alpha_{src}) \\
out_{RGB} = (src_{RGB} - dst_{RGB}) (\frac{\alpha_{src}}{255} ) + dst_{RGB}
\end{cases}
\end{equation}
 

Следует заметить, что под $out_{RGB}$  подразумевается кортеж значений $\langle R, G, B\rangle$, где $R, G, B \in [0, 255]$.

\section{Анализ существующих технологии оптимизации вычисления}
Оптимизация на низком уровне чаще всего дает больший прирост в скорости приложения, чем ручная оптимизация. Крупные компании-производетели процессоров, такие как Intel или AMD, работают над технологиями, позволяющими использовать определенные команды процессора для более эффективной работы с памятью. Также существует тенденция на перенос некоторых объемов вычислений на GPU. 

\subsection{Распараллеливание вычислений}
Распараллеливание вычислений предствляет огромный пласт знаний в области оптимизации алгоритмов. 

Идея распараллеливания вычислений основана на том, что большинство задач может быть разделено на набор меньших задач, которые могут быть решены одновременно. Характер увеличения скорости программы в результате распараллеливания объясняется законами Амдала и Густавсона. Обычно параллельные вычисления требуют координации действий.  

Параллельные вычисления использовались много лет в основном в высокопроизводительных вычислениях, но в последнее время к ним возрос интерес вследствие существования физических ограничений на рост тактовой частоты процессоров. Параллельные вычисления стали доминирующей парадигмой в архитектуре компьютеров, в основном в форме многоядерных процессоров.\cite{bib3}

Параллельные вычисления существуют в нескольких формах: 
\begin{enumerate}
\item Параллелизм на уровне битов. Основан на увеличении размера машинного слова. Увеличение размера машинного слова уменьшает количество операций, необходимых процессору для выполнения действий над переменными, чей размер превышает размер машинного слова.
\item Параллелизм на уровне инструкций является мерой того, какое множество операций в компьютерной программе может выполняться одновременно. 
\item Параллелизм данных или векторизация, заключается в том, что одна операция выполняется сразу над всеми элементами массива данных.
\item параллелизм задач, или параллелизм на уровне потоков, подразумевает, что вычислительная задача разбивается на несколько относительно самостоятельных подзадач и каждый процессор загружается своей собственной подзадачей.
\end{enumerate}

Писать программы для параллельных систем сложнее, чем для последовательных \cite{bib4}, так как конкуренция за ресурсы представляет новый класс потенциальных ошибок в программном обеспечении, среди которых состояние гонки является самой распространённой. Взаимодействие и синхронизация между процессами представляют большой барьер для получения высокой производительности параллельных систем.


\subsection{CUDA}
CUDA – это программно-аппаратная архитектура параллельных вычислений от NVIDIA, позволяющая существенно увеличить вычислительную производительность благодаря использованию GPU (графических процессоров) фирмы NVIDIA.

CUDA  позволяет программистам реализовывать на специальном упрощённом диалекте языка программирования Си алгоритмы, выполнимые на графических процессорах NVIDIA, и включать специальные функции в текст программы на Си. \cite{bib5} Архитектура CUDA даёт разработчику возможность по своему усмотрению организовывать доступ к набору инструкций графического ускорителя и управлять его памятью.

Разработчики программного обеспечения, ученые и исследователи широко используют CUDA в различных областях, включая обработку видео и изображений, вычислительную биологию и химию, моделирование динамики жидкостей, восстановление изображений, полученных путем компьютерной томографии, сейсмический анализ, трассировку лучей и многое другое.

Достоинствами CUDA является огромный прирост скорости выполнения расчётов по сравнению с расчетами на центральном процессоре компьютера. Для некоторых задач ускорение может измеряться сотнями секунд за счёт более эффективные транзакций между памятью центрального процессора и видеопамятью.

Недостатками является сложность программирования для CUDA (хотя производитель утверждает обратное), привязка к картам NVIDIA, и, как вследствие, отсутствие переносимости между архитектурами.


\subsection{SSE}
Семейство SSE (Streaming SIMD Extensions, потоковое SIMD-расширение процессора) — это SIMD (Single Instruction, Multiple Data, Одна инструкция — множество данных) наборы инструкций, разработанные Intel. На данный момент существует 4 поколения наборов. Каждое поколение принесло новые команды и увеличило производительность.\cite{bib2} 

\begin{enumerate}
	\item SSE. Включает в архитектуру процессора восемь 128-битных регистров и набор из 70 инструкций, работающих со скалярными и упакованными типами данных.
	\item SSE2. Расширяет SSE, добавляя 144 новых инструкции. Содержит инструкции для потоковой обработки целочисленных данных в тех же 128-битных xmm регистрах, что делает это расширение более предпочтительным для целочисленных вычислений. Включает в себя ряд команд управления кэшем, предназначенных для минимизации загрязнения кэша при обработке объёмных потоков данных, а также сложные дополнения к командам преобразования чисел.
	\item SSE3. Добавляет 13 новых команд, состоящих из команды сложения и вычитания нескольких значений, хранящихся в одном регистре и новой команды для преобразования значений с плавающей точкой в целые без необходимости вносить изменения в глобальном режиме округления.
	\item SSE4. Состоит из 54 инструкций, 47 из них относят к SSE4.1. Ни одна из SSE4 инструкций не работает с 64-битными mmx регистрами (только со 128-битными xmm0-15).
\end{enumerate}

\begin{figure}[ht!]
	\centering{ 
		\includegraphics[width=1\textwidth]{img/img8.png}
		\caption{Регистры xmm0 -- xmm7}}
\end{figure}

Преимущество в производительности достигается в том случае, когда необходимо произвести одну и ту же последовательность действий над разными данными. В таком случае блоком SSE осуществляется распараллеливание вычислительного процесса между данными.

Приложение, где одно и то же значение добавляется к (или вычитается из) большое набора данных, может использовать преимущества SSE: обычная операция во многих мультимедийных приложений.

У данной технологии есть несколько преимуществ:
\begin{itemize}
	\item Загружаются в память сразу блок данных, что по многим причинам быстрее, чем последовательная загрузка.
	\item Операция выполняется над всем блоком данных одновременно в одной операции.
\end{itemize}

Стоит также упомянуть несколько недостатков SSE:
\begin{itemize}
	\item Не все алгоритмы могут быть легко векторизованы.
	\item В настоящее время, реализация алгоритма с инструкциями SSE, как правило, требует человеческого труда. Большинство компиляторов не генерируют SSE инструкции из программы на C/C++.
	\item SSE может иметь ограничения на выравнивание данных, что требует дополнительных усилий со стороны программиста.
	\item Некоторые наборы ограничены определенной архитектурой процессора.
\end{itemize}



\subsection{AVX}
Advanced Vector Extensions (AVX) являются расширениями x86 набора инструкций архитектуры для микропроцессоров от Intel и AMD\cite{bib2}.  AVX2 расширяет большинство целых команд до 256 бит и вводит операции плавного многократного накопления (FMA), а также новые инструкции и новую схему кодирования машинных кодов.

AVX использует шестнадцать регистров YMM0--YMM15 вместо XMM0--XMM15. AVX вводит трехоперандный формат команды SIMD, где регистр назначения отличен от двух исходных операндов, сохраняя тем самым оба исходных операнды.
Упрощено требование выравнивания операндов памяти\cite{bib2}. Новая схема кодирования VEX вводит новый набор кодов префиксов, который расширяет пространство кодов операндов, позволяет командам иметь более двух операндов, и SIMD векторому регистру быть больше, чем 128 бит. Приставка VEX также может быть использована на унаследованных SSE инструкциях, давая им трехоперандную форму, что делает более эффективным взаимодействие с инструкциями AVX.

Advanced Vector Extensions 2 (AVX2), также известный как Haswell New Instructions \cite{bib6},  является расширением набора инструкций AVX, введенной компанией Intel в Haswell микроархитектуре. AVX2 делает следующие дополнения:
\begin{itemize}
\item Pасширение векторных целочисленных SSE и AVX инструкций до 256 бит
\item Трехопрандные операции с битами общего назначения
\item Поддержка загружки векторных элементов из несмежных областей памяти
\item DWORD- и QWORD-гранулярность любых переменных
\item Векторные смещения
\end{itemize}

AVX-512 являются 512-разрядными расширениями к 256-битным Advanced Vector Extensions SIMD инструкциям для x86 архитектуры, предложенной Intel в июле 2013 года. Инструкции AVX-512 кодируются с новым префиксом EVEX . Это предоставляет использование 4-х операндов, 7 новых 64-битных opmask регистров, скалярный режим памяти с автоматической трансляцией, явным контролем округления,режим адресации памяти со сжатым смещением. Ширина регистра увеличивается до 512 бит и общее количество регистров увеличилось до 32 (регистры ZMM0-ZMM31) в режиме x86-64.

AVX применяется в мультимедиа, научных и финансовых приложений (AVX2 добавляет поддержку для целочисленных операций).

Преимущества перед другими технологиями:
\begin{itemize}
\item Увеличивает параллельность и пропускную способность в плавающей запятой SIMD вычислений.
\item Снижает нагрузку на регистр вследствие неразрушающих инструкций.
\item Поддержка AVX реализована в следующих популярных компиляторах:
Microsoft C/C++ Compiler начиная с версии 16 (входит в Visual Studio 2010);
Intel C++ Compiler начиная с версии 11.1;
GCC начиная с версии 4.4;
\end{itemize}

Однако, помимо "унаследованных" недостатков SSE, AVX имеет пару своих недостаков:
\begin{itemize}
	\item Смешивание неадаптированных SSE и AVX инструкций приведёт к заметному снижению производительности, т.к. при переходе процессор сохраняет или восстанавливает в специальном кэше верхние 128 бит AVX регистров, на что уходит около сотни тактов. Стоит использовать SSE в префиксом VEX/EVEX или команды vzeroupper или vzeroall.
	\item Сложная переносимость кода. Код обрастает условными директивами. 
\end{itemize}
