\chapter{ Констукторский раздел}
\label{cha:design}

За основу возьмем открытый графический редактор KDE Krita (https://krita.org/en/). Основная логика операций с цветами заложена в шаблонном классе KoColorSpaceMaths в одноименных файлах. 

\lstset{breaklines=true, numbers=left,
	keywordstyle=\color{blue}, commentstyle=\color{gray}}
\begin{lstlisting}[language=c++]
/* _T -- некоторый целочисленный тип c известной спецификацией 
* (миниальное/максимальное значение, начальное, нулевое значения, 
* эпсилон-значение, длина в битах), которая должна быть определена 
* в шаблонном классе KoColorSpaceMathsTraits через его явное определение.  
* _Tdst -- выходной тип, необходим,  если выходной тип не совпадает
*  с входным, иначе они равны */
template < typename _T, typename _Tdst = _T >
class KoColorSpaceMaths
{
    typedef KoColorSpaceMathsTraits<_T> traits;
    typedef typename traits::compositetype src_compositetype;
    typedef typename KoColorSpaceMathsTraits<_Tdst>::compositetype dst_compositetype;
public:
	// Перемножение двух чисел
    inline static _Tdst multiply(_T a, _Tdst b);
    // Перемножение трех чисел
    inline static _Tdst multiply(_T a, _Tdst b, _Tdst c);
    // Деление
    inline static dst_compositetype divide(_T a, _Tdst b);
    // Инверсия числа = начальное значение - текущее
    inline static _T invert(_T a);
    // 
    inline static _T blend(_T a, _T b, _T alpha);
    inline static _Tdst scaleToA(_T a);
    inline static dst_compositetype clamp(dst_compositetype val);
    inline static _Tdst clampAfterScale(dst_compositetype val);
};
\end{lstlisting}

%%% Local Variables:

%%% mode: latex
%%% TeX-master: "rpz"
%%% End: