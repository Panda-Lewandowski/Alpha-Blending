\chapter{ Технологический раздел}
\label{cha:design}

Проведем оптимизацию смешения цветов в открытом графическом редакторе KDE Krita (https://krita.org/). 


\section{Выбор вспомогательных библиотек}
Текущие компиляторы C++ могут выполнять автоматическое преобразование скалярных программ в инструкции SIMD (автовекторизация). Однако компилятор должен свойства алгоритма, который мог быть потерян, когда разработчик написал чисто скалярную реализацию в C++. Следовательно, компиляторы C++ не могут векторизовать какой-либо данный код в его наиболее эффективный параллельный вариант данных. Особенно большие параллельные данные, охватывающие несколько функцийи более, часто не будут преобразованы в эффективный код SIMD.

Бибилиотека Vc помгает справится с этой задачей. Ее типы позволяют явно указывать параллельные операции с данными по нескольким значениям. Поэтому параллелизм добавляется через систему типов. Конкурирующие подходы определяют параллелизм через новые структуры управления и, следовательно, новую семантику внутри тела этих структур управления.

Vc - это бесплатная библиотека программного обеспечения для упрощения явной векторизации кода на C++. Она имеет интуитивно понятный API и обеспечивает переносимость между различными компиляторами и версиями компилятора, а также переносимость между различными наборами векторных команд. 
\cite{bib8}

\newpage
\section{Необходимые типы данных и классы}

\begin{table}[h!]
	\begin{center}
		\begin{tabular}{|c|c|}
			\hline
			Тип / Класс& Описание \\
			\hline
			\multicolumn{2}{|c|}{Библиотека Vc} \\
			\hline
			Vc::uint16\_v & \specialcell{$typedef\; Vector<std::uint16\_t>$. \\ Класс Vector является базовым в Vc. \\Имеет фиксированное количество \\ элементов.} \\
			\hline
		   Vc::Implementation & \specialcell{Перечисление, опредяющее определенный \\ набор команд SIMD.}\\
		   \hline
			Vc::SimdArray & \specialcell{Используется как $unt32\_16\_v, int\_v,$ \\ $ uint16\_16\_v, uint\_v$. \\ Параллельный арифметический тип \\ данных с задаваемым  количеством\\  элементов.}\\
			\hline
			\multicolumn{2}{|c|}{Krita} \\
			\hline
			KoStreamedMath&  \specialcell{Структура, определяющая набор \\ необходимых  функций \\ для реализации алгоритмов \\ работы с цветом.}\\
			\hline
			OptimizedOverCompositor32 &\specialcell{Оптимизированная версия KoCompositeOp\\ для использования в 4 байтовых цветовых \\ пространствах с альфа-каналом, \\ размещенным в последнем байте пикселя. }\\
			\hline
			 KoDoubleOptimizedCompositeOpOver32 & \specialcell{Класс, реализующий идею \\ описанную выше.\\ Также наследуется от KoCompositeOp.}\\
			\hline
		\end{tabular}
	\end{center}
\end{table} 

\section{KoStreamMath}
Основная логика работы с цветами происходит в модуле KoStreamMath. Чтобы определить новый класс, который позволит нам внедрить оптимизированное смешение в проект, напишем несколько основных функций, реализующих логику работы с каналами и цветами.

\subsection{Распаковка маски}
Функция получает на вход вектор, содержащий первые Vc::uint16\_v::size() значений маски и упаковывает их в вектор c помощью конструктора  Vc::uint16\_v. Функция является статичной, встраиваемой в код (inline).

\begin{lstlisting}[language=c++]
static inline Vc::uint16_v fetch_mask_8_uint16(const quint8 *data) {
    Vc::uint16_v data_i(data);
    return data_i;
}
\end{lstlisting}

\subsection{Распаковка альфа-канала}
Шаблонная функция, распаковывает значения прозрачности в вектор в зависимости от выравнивания и сдвигает в последний байт пикселя. Контролировать выравнивание необходимо по двум причнам:
\begin{enumerate}
\item Получение выровненных данных с невыровненной инструкцией ухудшает производительность.  
\item Получение невыровненных данных с выровненной инструкцией вызывает \#GP (исключение общей защиты).
\end{enumerate}

\begin{lstlisting}[language=c++]
template <bool aligned>
static inline uint16_16_v fetch_alpha_uint16(const quint8 *data) {
    uint32_16_v data_i;
   if (aligned) {
       data_i.load((const quint32*)data, Vc::Aligned);
   } else {
       data_i.load((const quint32*)data, Vc::Unaligned);
   }

   return uint16_16_v(data_i >> 24);
}
\end{lstlisting}

\subsection{Распаковка цветовых каналов}
Аналогично предыдущим функциям-распаковкам. 
\begin{lstlisting}[language=c++]
template <bool aligned>
static inline void fetch_colors_uint16(const quint8 *data,
                            Vc::uint16_v &c1,
                            Vc::uint16_v &c2,
                            Vc::uint16_v &c3) {
    Vc::uint32_v data_i;
    if (aligned) {
        data_i.load((const quint32*)data, Vc::Aligned);
    } else {
        data_i.load((const quint32*)data, Vc::Unaligned);
    }

    const quint32 lowByteMask = 0xFF;
    Vc::uint32_v mask(lowByteMask);

    c1 = Vc::uint16_v((data_i >> 16) & mask);
    c2 = Vc::uint16_v((data_i >> 8)  & mask);
    c3 = Vc::uint16_v(data_i         & mask);
}
\end{lstlisting}

\subsection{Упаковка каналов}
Пакует цвет и альфа-значения в 4 канала по 8 бит на канал. Данные цвета хранятся в 3-х наименее значимых байтах пикселя, альфа - в наиболее значимом.

\begin{lstlisting}[language=c++]
static inline void write_channels_uint16(quint8 *data,
                                     Vc::uint16_v::AsArg alpha,
                                     Vc::uint16_v::AsArg c1,
                                     Vc::uint16_v::AsArg c2,
                                     Vc::uint16_v::AsArg c3) {
    const quint32 lowByteMask = 0xFF;

    uint32_16_v mask(lowByteMask);
    uint32_16_v v1 = uint32_16_v(Vc::round(alpha)) << 24;
    uint32_16_v v2 = (uint32_16_v(Vc::round(c1)) & mask) << 16;
    uint32_16_v v3 = (uint32_16_v(Vc::round(c2)) & mask) <<  8;
    uint32_16_v v4 = uint32_16_v(Vc::round(c3))  & mask;
    v1 = v1 | v2; 
    v3 = v3 | v4;
    (v1 | v3).store((quint32*)data, Vc::Aligned);
}
\end{lstlisting}

%Dmitry Kazakov, [Nov 26, 2017, 6:58:13 PM]:
%Я бы начал с Баррета, а потом сказал бы, что мы просто добавляем округление к %операциям деления

%Плюс сравнил бы результат при умножении 0, 127, 128 и 255 между собой

%Эти результаты подсказали бы, что к Баррету нужно добавить округление

%До формулы 2.2 все норм, а потом должен начаться Баррет.

\subsection{Умножение}

\begin{lstlisting}[language=c++]
static inline Vc::uint16\_v optimizedVectorMultiply(Vc::uint16\_v a, Vc::uint16\_v b)
{
static const Vc::uint16\_v offset(0x80u);
Vc::uint16\_v c = a * b + offset;
return ((c >> 8) + c) >> 8;
}
\end{lstlisting}

Операция умножения компонент RGB8 и множество всех значений от 0 до 255 составляют собой некоммутативный моноид. В первую очередь, это означает, что $\forall a, b : a * b \in [0, 255]$, где знаком $*$ представлена операция умножения, то есть выполняется умножению по модулю 255. 

\begin{equation}
result = a * b~mod~255
\end{equation}


Следует заметить, что оптимизации деления на 255 не существует. Можно воспользоваться тем, что деление на 256 есть не что иное, как побитовый сдвиг вправо на 8. Однако стоит помнить, что должны использоваться целые числа и 
\begin{equation}
\frac{255*255}{256} \approx 254.0039
\end{equation}

Далее должно произойти округление до 255, т.к. 255 есть нейтральный элемент в \textit{моноиде RGB8}.
Следовательно, необходимо использовать некое округление, причем можно найти случаи, когда округление ведется не в большую сторону, как в примере (3.2).

\begin{equation}
\frac{100*200}{256} \approx 78,125
\end{equation}

Хотя 

\begin{equation}
\frac{100*200}{255} \approx 78,4314 \approx 78
\end{equation}

Здесь важна также одна небольшая деталь: при побитовом сдвиге число, над которым будет производиться данная операция, приводиться в типу \textit{integer} или производным от него, то есть дробная часть \textit{отбрасывается}. При округлении нам необходимо либо превысить исходное значение (тогда округление произойдет в большую сторону), либо не выходить за его пределы (тогда округление произойдет в меньшую сторону). 

Добавим к умноженному значению число $t = \frac{256}{2}$, которое будет "приближать число" в сторону большего значения, и если число имело часть меньшую 0.5, то увеличение не произойдет.

Однако, в некоторых случаях умножение по модулю будет до сих пор происходить неверно: $(255*255+128) >> 8 = 254$. Добавим еще одно корректирующие значение. Только теперь оно будет зависить от полученного результата и будет дополнять число до нужного.

Получим формулу из листинга: 
\begin{equation}
\begin{cases} 
c = (a * b + 128) >> 8 \\
result = ((c >> 8) + c) >> 8
\end{cases}
\end{equation}

Доказательство данных  предположений можно вывести благодаря работе Барретта, представившего в 1986 году  "Сокращение Барретта", алгоритма быстрого вычисления числа по модулю \cite{bib7}.


\subsection{Деление}
По такой же схеме можно вывести следующее:
\begin{lstlisting}[language=c++]
static inline Vc::uint16\_v optimizedVectorDevide(Vc::uint16_v a, Vc::uint16_v b)
{
static const Vc::uint16_v part(2u);
Vc::uint16_v c = (a * UINT8_MAX + (b / part)) / b;
return c;
}
\end{lstlisting}

\subsection{Смешение}
Следует учесть, что в Крите используется пряма альфа по причинам использования масок и прозрачности слоя, поэтому в данном контексте она смотрится удачнее премультиплексированной. Однако стоит вынести нормализацию по новой $\alpha$ за рамки смешения и провести ее с помощью функции деления описанной выше. таким образом, данная функция реализует формулу $a*alpha + b*(1-alpha)$
\begin{lstlisting}[language=c++]
static inline Vc::uint16_v optimizedVectorBlend(Vc::uint16_v a, Vc::uint16_v b, Vc::uint16_v alpha)
{
static const Vc::uint16_v offset(0x80u);
Vc::uint16_v c = (a - b) * alpha + offset;
c = ((c >> 8) + c) >> 8;
return c + b;
}

\end{lstlisting}


%%% Local Variables:
%%% mode: latex
%%% TeX-master: "rpz"
%%% End:
